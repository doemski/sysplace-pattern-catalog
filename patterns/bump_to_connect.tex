\input{header.tex}

\newcommand{\name}{Bump To Connect}

% -------------------------------
% WAS
% -------------------------------

\newcommand{\desc}{Ein Benutzer (der \sen{}) möchte sein Gerät (das \sendev{}) durch direkten Kontakt mit einem anderen Gerät (dem \recdev{}) verbinden um eine Datenübertragung zu ermöglichen.}

\newcommand{\solution}{Der \sen\ hat das \sendev{} in der Hand. Durch das  zusammenstoßen (\glslink{bump}{Bumpen}) des \sendev s\ mit dem \recdev{} wird der Verbindungsvorgang gestartet.}

%\newcommand{\category}{give}
%\newcommand{\category}{take}
%\newcommand{\category}{exchange}
%\newcommand{\category}{extend}
\newcommand{\category}{connect}

% -------------------------------
% WIE
% -------------------------------

\newcommand{\useraction}{Der \sen\ hält das \sendev\ in der Hand und stößt es leicht an das \recdev\ an. Das \recdev\ kann dabei von einer weiteren Person gehalten werden oder stationär sein (z.B ein Tablet oder Tabletop).}

\newcommand{\reactionSen}{Auf dem \sendev\ sollte zu verschiedenen Phasen des Bumps Feedback gegeben werden. Ein Bump besteht aus den drei \glslink{atomareinteraktion}{atomaren Interaktionen} \textit{Move}, \textit{Abrupt Stop} und \textit{Bump Recognized}.\\
Da es sich bei \textbf{Bump to Connect} um eine \gls{synchronegeste} handelt, werden nach dem Bump die \glslink{accelerometer}{Accelerometerdaten} der beteiligten Geräte verglichen. Vor dem Bump besteht keine direkte Verbindung zwischen den Geräten, daher muss es eine \gls{vermittlungskomponente} geben (z.B. ein externer Server), an die die Daten zur Überprüfung gesendet werden. Sind die Daten ähnlich genug, gilt der Bump als \textit{erfolgreich}. Bei einem \textit{erfolgreich} ausgeführten Bump finden alle atomaren Interaktionen statt, nach Beendigung der letzten werden die Geräte drahtlos verbunden (Atomare Interaktion \textit{Connect}), worüber der Nutzer ebenfalls Feedback erhalten sollte.}

\newcommand{\reactionRec}{Handelt es sich bei dem \recdev\ um ein mobiles Gerät, das von einem Nutzer gehalten wird, werden die selben atomaren Interaktionen ausgeführt wie auf dem \sendev{}.
}

\newcommand{\microinteractionstabular}{
\begin{figure}[H]
\begin{table}[H]
\renewcommand{\arraystretch}{2}\addtolength{\tabcolsep}{-2pt}
\centering
\newcolumntype{b}{X}
\newcolumntype{t}{>{\hsize=.3\hsize}X}
\newcolumntype{s}{>{\hsize=.2\hsize}c}
\newcolumntype{m}{>{\hsize=.6\hsize}X}
\begin{tabularx}{\textwidth}{tsbbm}
\thead[X]{Name} & \thead[c]{Typ*} & \thead[X]{Trigger} & \thead[X]{Regeln} & \thead[X]{Feedback} \\
\hline
Move & M & Gerät wurde bewegt (Accelerometerdaten) & \data\ ist ausgewählt &  Animation 1 \\ 
\hline
Abrupt Stop & M & Das Gerät ist mit etwas hartem zusammengestoßen (Accelerometerdaten) & Accelerometerdaten erfüllen Bump-Anforderungen & Animation 2 \\ 
\hline
Bump Recognized & S & Bump-Daten an Vermittlungskomponente gesendet und Antwort erhalten & Accelerometerdaten beider Geräte ähnlich genug & Animation 3 \\ 
\hline
Connect & S & Verbindung zwischen den Geräten wurde hergestellt & Datentransfer \newline ist möglich & Animation 4 \\
\hline
\end{tabularx}
\end{table}
\caption{Atomare Interaktionen für das Bump to Connect Pattern}
\end{figure}
*Typ: (M)anuell, (S)ystem
}

\newcommand{\animations}{
\begin{enumerate}
\item Move-Animation: visualisiert dem Benutzer, dass das physische Bewegen des Geräts eine Funktionalität darstellt (z.B. Ein Objekt auf dem Bildschirm bewegt sich etwas verzögert, als hätte es Masse)
\item Abrupt Stop-Animation: visualisiert dem Benutzer, dass ein Bump auf seinem Gerät erkannt wurde (z.B. Das Objekt bleibt am angestoßenen Rand des Bildschirms)
\item Bump Recognized-Animation: visualisiert dem Benutzer, dass die Bump-Geste richtig erkannt wurde. (z.B. Das Objekt färbt sich grün)
\item Connect-Animation: visualisiert dem Benutzer, dass eine Verbindung hergestellt wurde(z.B. Das Objekt wird zu einem Stecker und bewegt sich in Richtung des anderen Geräts)
\end{enumerate}
}

\newcommand{\designnotes}{
\begin{itemize}
\item[-] Beim Ausführen der Bump-Geste sollten die Geräte direkt aneinandergestoßen werden um eine möglichst gute Erkennung zu gewährleisten. Zudem sollte die Härte des Zusammenstoßes stark genug sein um eine Erkennung zu ermöglichen, jedoch die Geräte nicht beschädigen.
\item[-] Bei jeder Synchronen Connect-Geste muss es eine Vermittlungskomponente geben, an die die relevanten Daten der Geste gesendet werden. Diese vergleicht die empfangenen Daten der beiden involvierten Geräte und gibt positive oder negative Rückmeldung.
\end{itemize}}

% -------------------------------
% WANN
% -------------------------------

\newcommand{\validcontext}{Verbinden von mobilen Geräten zwecks Datenübertragung, Verbinden von mobilen Privatgeräten mit stationären Geräten (z.B. Verbindung zu einem Netzwerk am Arbeitsplatz)}

\newcommand{\simultaneously}{}
%\newcommand{\sequentially}{}

\newcommand{\online}{}
%\newcommand{\offline}{}

\newcommand{\private}{}
\newcommand{\semipublic}{}
\newcommand{\public}{}
\newcommand{\stationary}{}
\newcommand{\onthego}{}

\newcommand{\leanback}{}
\newcommand{\leanforward}{}

\newcommand{\single}{}
\newcommand{\collaboration}{}
\newcommand{\facetoface}{}
\newcommand{\sidetoside}{}
\newcommand{\cornertocorner}{}

\newcommand{\notvalidcontext}{Sichtbarmachen vertraulicher Informationen (z.B. Name, Alter etc.) auf öffentlichen Displays.}


\newcommand{\devicetabular}{
\begin{tabular}[H]{|c|c|c|c|c|c|}
\hline 
\diagbox{von}{nach}   & Smartwatch & Smartphone & Tablet & Tabletop & Screens \\ 
\hline 
Smartwatch            &     x      &     x      &   x    &    x     &         \\ 
\hline 
Smartphone            &     x      &     x      &   x    &     x    &        \\ 
\hline 
Tablet                &     x      &     x      &   x    &     x    &    
\\ 
\hline 
Tabletop              &            &            &        &          &         \\ 
\hline
Screens               &            &            &        &          &         \\ 
\hline 
\end{tabular}}

% -------------------------------
% WARUM
% -------------------------------

%\newcommand{\established}{}
\newcommand{\candidate}{}
\newcommand{\realizable}{}
%\newcommand{\futuristic}{}

\newcommand{\otherpatterns}{
\begin{itemize}
\item Bump To Give
\item Bump To Take
\item Bump To Exchange
\item Nudge
\end{itemize}
}

\newcommand{\stateoftheart}{
\begin{enumerate}
\item Bump App: Bis 2014 in den App/Play Stores erhältlich gewesen [\url{http://bu.mp/}]
\item Beispiel-Implementierung eines Bump Patterns \citep{Grab2015}.
\item Ken Hinckley - Forschung zu Bump-Gesten im Speziellen und Definition synchroner Gesten im Allgemeinen \citep{Hinckley2003}.
\end{enumerate}
}

\newcommand{\designprinciples}{}

\newcommand{\imageschemata}{}
%\newcommand{\imageSchemaVoid}{}
%\newcommand{\imageSchemaObject}{}
%\newcommand{\imageSchemaSubstance}{}
%\newcommand{\imageSchemaCenterPeriphery}{}
\newcommand{\imageSchemaContact}{}
%\newcommand{\imageSchemaFrontBack}{}
%\newcommand{\imageSchemaLocation}{}
%\newcommand{\imageSchemaNearFar}{}
%\newcommand{\imageSchemaPath}{}
%\newcommand{\imageSchemaSourcePathGoal}{}
%\newcommand{\imageSchemaScale}{}
%\newcommand{\imageSchemaLeftRight}{}
%\newcommand{\imageSchemaContainer}{}
%\newcommand{\imageSchemaContent}{}
%\newcommand{\imageSchemaFullEmpty}{}
%\newcommand{\imageSchemaInOut}{}
%\newcommand{\imageSchemaSurface}{}
%\newcommand{\imageSchemaMerging}{}
%\newcommand{\imageSchemaSplitting}{}
\newcommand{\imageSchemaMomentum}{}
%\newcommand{\imageSchemaSelfMotion}{}
%\newcommand{\imageSchemaBigSmall}{}
%\newcommand{\imageSchemaFastSlow}{}
%\newcommand{\imageSchemaPartWhole}{}

\newcommand{\realworld}{}
\newcommand{\realworldNaivePhysic}{}
\newcommand{\realworldBodyAwareness}{}
\newcommand{\realworldEnvironmentAwareness}{}
\newcommand{\realworldSocialAwareness}{}

\newcommand{\metaphor}{}
\newcommand{\metaphordesc}{Begrüßung per "Fistbump", Anstoßen (wie mit Getränken)}

% -------------------------------
% TECHNISCHES
% -------------------------------

\newcommand{\requiredTechnologies}{
Um Bump To Connect auf einem Gerät (\textit{Device}) ausführen zu können, gibt es einige Voraussetzungen und Einschränkungen bezüglich der in dem Gerät verfügbaren Technologien. Ein Gerät ist dann verwendbar, wenn es folgende Eigenschaften aufweist (s. Abbildung \ref{bump_hardware}):
\begin{itemize}
\item \textbf{Input}: Unbedingte Voraussetzung für die Erkennung von Bump-Gesten ist die Möglichkeit, die Beschleunigung eines Gerätes messen zu können. Dazu wird in mobilen Geräten zumeist ein \Gls{accelerometer} verbaut, dessen Werte von Applikationen verarbeitet werden können.
\item \textbf{Output}: Je nachdem, welches Feedback dem User gegeben werden soll, sind visueller Output (Bildschirm) sowie Vibration und Sound einsetzbar, um erkannte Geräte und aufgebaute Verbindungen zu signalisieren.
\item \textbf{Connectivity}: Ziel der Bump To Connect Geste ist das Aufbauen einer Verbindung zwischen zwei Geräten. Da die Geräte sich vorher nicht kennen, aber ein Protokoll für die Erkennung der synchronen Geste verwendet wird, ist eine \Gls{vermittlungskomponente} (z.B. Message-Queue-Server, Webservice o.Ä.) zur Vermittlung zwischen den Geräten notwendig. Der anschließende Verbindungsaufbau zwischen den Geräten erfolgt über ein Ad-Hoc Netzwerk wie Bluetooth oder WiFi Direct.
\end{itemize}

\begin{figure}[h]
\includegraphics[width=\textwidth]{bump_hardware.png}
\caption{Benötigte Technologien für das Bump To Connect Pattern}
\label{bump_hardware}
\end{figure}
}

\newcommand{\implementation}{
\subsubsection*{Erkennung synchroner Gesten}
Bei der Geste Bump To Connect handelt es sich um eine \gls{synchronegeste}. Abbildung \ref{sync_gesture_detection} zeigt den Ablauf der synchronen Gestenerkennung, die in mehrere Teile zerfällt:

\begin{figure}[h]
\includegraphics[width=\textwidth]{synchronous_gesture_detection.png}
\caption{Allgemeiner Ablauf einer synchronen Gestenerkennung}
\label{sync_gesture_detection}
\end{figure}

\begin{itemize}
\item Registrierung bei einer zentralen Vermittlungskomponente,
\item Erkennung der Einzelanteile einer synchronen Geste pro Gerät (\textit{Gesture Detection)}, entspricht jeweils der Erkennung für \glspl{einfachegeste},
\item Vergleich der resultierenden \textit{GestureEvents} aus der einfachen Gestenerkennung (\textit{Gesture Event Matching}) durch die Vermittlungskomponente und
\item Vermittlung von Verbindungsdaten, wenn die Geste erfolgreich erkannt wurde.
\end{itemize}

Nach Registrierung bei der zentralen Vermittlungskomponente führen Geräte kontinuierlich eine lokale, einfache Gestenerkennung basierend auf Sensordaten (Hard- oder Software) durch (\textit{Gesture Detection)}. Wurde  eine Geste lokal erkannt und an das entsprechende \textit{GestureEvent} an die Vermittlungskomponente gesendet, entscheidet diese, ob alle erforderlichen komplementären Anteile der synchronen Geste durch andere Geräte ebenfalls übermittelt wurden (\textit{Gesture Event Matching}). Die folgenden beiden Kapitel erläutern diese beiden Schritte für die Bump To Connect Geste jeweils im Detail.

\subsubsection*{Bump Gesture Detection}
Die Gesture Detection für Bump Gesten speichert eine Reihe von Beschleunigungsvektoren des Accelerometers, sobald ein anfänglicher Schwellwert (Parameter \texttt{threshold}) überschritten wurde (Übergang vom state \texttt{idle} in den state \texttt{recording}). Die Anzahl der gespeicherten Beschleunigungsvektoren vor dem Übergang in den state \texttt{Counting Peaks} hängt vom Parameter \texttt{win\-dow\_size} ab. Im state \texttt{Counting Peaks} werden sog. \texttt{peaks} (Spitzen) in der Beschleunigung gezählt, also Beschleunigungsvektoren, deren Betrag jeweils größer oder kleiner als die der nächsten Nachbarn in der Liste ist. Ist die Anzahl der \texttt{peaks} größer als der Parameter \texttt{min\_peaks}, wird ein Bump erkannt und an die Vermittlungskomponente übermittelt (s. Abbildung \ref{sync_gesture_detection}).

\begin{figure}[H]
\includegraphics[width=\textwidth]{bump_recognize.png}
\caption{Gesture Detection der Bump Geste lokal auf einem Gerät}
\label{recognize_bump}
\end{figure}

Über die drei Parameter \texttt{threshold}, \texttt{min\_peaks} sowie \texttt{window\_size} lässt sich die Empfindlichkeit der Bumperkennung einstellen bzw. Ungenauigkeiten bei der Verwendung unterschiedlicher Accelerometer ausgleichen.

\subsubsection*{Bump Event Matching}
Die Vermittlungskomponente führt ein \textit{Bump Event Matching} durch, wenn von zwei Geräten \textit{BumpEvents} empfangen wurden, die als komplementäre Anteile einer synchronen Geste in Frage kommen könnten. Abbildung \ref{bump_event_matching} veranschaulicht einen exemplarischen Ablauf dieses Vergleiches.

\begin{figure}[H]
\includegraphics[width=\textwidth]{bump_event_matching.png}
\caption{Bump Event Matching - Vergleichen zweier Bump Events durch die Vermittlungskomponente}
\label{bump_event_matching}
\end{figure}

Grundsätzlich kann der Vergleich anhand beliebiger Parameter vorgenommen werden, die im übermittelten \textit{BumpEvent} enthalten sind. Das Beispiel in Abbildung \ref{bump_event_matching} vergleicht zwei Events anhand ihrer
\begin{itemize}
\item höchsten Peaks (\texttt{max\_peak\_a} bzw. \texttt{max\_peak\_b}) mit einer konfigurierbaren Toleranz von \texttt{max\_peak\_delta} und
\item ihrer Zeitstempel (\texttt{timestamp\_end\_a} und \texttt{timestamp\_end\_b}) mit einer  konfigurierbaren Toleranz von \texttt{max\_time\_delta}.
\end{itemize} 

Weitere Vergleiche sind denkbar, sofern die \textit{Bump Gesture Detection} diese Werte im \textit{BumpEvent} zur Verfügung stellt.

Entsprechend dem Lebenszyklus einer Multiscreen-Applikation besteht nach erfolgreich
ausgeführtem Bump To Connect eine Verbindung zwischen zwischen zwei Geräten, was die Voraussetzung für anschließende Transfer- oder Disconnect-Gesten ist.

Weitergehende Informationen zum Applikations-Lebenszyklus und den weiteren
Gestaltungsmöglichkeiten für den \textit{Connect}, \textit{Feedbacks} etc. finden sich
auf der \developerpage.
}

% -------------------------------
% SONSTIGES
% -------------------------------

\newcommand{\authors}{
Benjamin Grab, Hochschule Mannheim\\
Valentina Burjan, Hochschule Mannheim\\
Dominick Madden, Hochschule Mannheim\\
Horst Schneider, Hochschule Mannheim}

\newcommand{\versionhistory}{10.05.2017}
\newcommand{\dateofcreation}{20.09.2015}


% template inkludieren --------------

\begin{document}

% ------ fixes the build for all patterns where those new variables haven't been defined yet

%\ifdefined\reactionSen
%\else
%\newcommand{\reactionSen}{tbd.}
%\fi
	
%\ifdefined\reactionRec
%\else
%\newcommand{\reactionRec}{tbd.}
%\fi

%\ifdefined\microinteractionstabular
%\else
%\newcommand{\microinteractionstabular}{tbd.}
%\fi

%\ifdefined\animations
%\else
%\newcommand{\animations}{tbd.}
%\fi

%\ifdefined\requiredTechnologies
%\else
%\newcommand{\requiredTechnologies}{tbd.}
%\fi

%\ifdefined\implementation
%\else
%\newcommand{\implementation}{tbd.}
%\fi

\maketitle

%----------------------------
% CATEGORY ICON
%----------------------------
\begin{textblock}{2}[0,0](8, -3)
\ifthenelse{\equal{\category}{give}}{\newcommand{\icon}{icon_give.png}}{}
\ifthenelse{\equal{\category}{take}}{\newcommand{\icon}{icon_take.png}}{}
\ifthenelse{\equal{\category}{connect}}{\newcommand{\icon}{icon_connect.png}}{}
\ifthenelse{\equal{\category}{extend}}{\newcommand{\icon}{icon_extend.png}}{}
\ifthenelse{\equal{\category}{exchange}}{\newcommand{\icon}{icon_exchange.png}}{}	
\includegraphics[scale=0.8]{\icon}
\end{textblock}

% -------------------------------
% WAS
% -------------------------------
\section*{Was}

\subsection*{Problem}
\desc

\subsection*{Lösung}
\solution

\subsection*{Grafische Darstellung}
\begin{figure}[H]
\IfFileExists{\jobname_graphical_description.png}{\includegraphics[width=\textwidth]{\grafischedarstellung}}{}
\end{figure}

\subsection*{Kategorie}
\ifthenelse{\equal{\category}{give}}{$\boxtimes$}{$\Box$} Give   |   
\ifthenelse{\equal{\category}{take}}{$\boxtimes$}{$\Box$} Take   |   
\ifthenelse{\equal{\category}{exchange}}{$\boxtimes$}{$\Box$} Exchange   |   
\ifthenelse{\equal{\category}{extend}}{$\boxtimes$}{$\Box$} Extend   |   
\ifthenelse{\equal{\category}{connect}}{$\boxtimes$}{$\Box$} Connect

% -------------------------------
% WIE
% -------------------------------
\newpage
\section*{Wie}

%%%define swapReactions if the receiver is the primary actor%%%
%%%define usersEqual if there is no sender and receiver%%%
\ifx\reactionSen\undefined
\else
	\ifx\swapReactions\undefined
		\ifx\usersEqual\undefined	
			\subsection*{Aktion des \sen s}
			\useraction

			\subsection*{Reaktionen des \sendev s}
			\reactionSen

			\subsection*{Reaktionen des \recdev s}
			\reactionRec
		\else
			\subsection*{Aktion der Benutzer}
			\useraction

			\subsection*{Reaktionen der Geräte}
			\reactionSen
		\fi
	\else
		\subsection*{Aktion des \rec s}
		\useraction
	
		\subsection*{Reaktionen des \recdev s}
		\reactionRec
	
		\subsection*{Reaktionen des \sendev s}
		\reactionSen
	\fi
\fi	

\ifx\microinteractionstabular\undefined
\else
	\subsection*{Übersicht über die Atomaren Interaktionen}
	\microinteractionstabular
\fi

\ifx\animations\undefined
\else
	\subsubsection*{Animationen}
	\animations
\fi

\subsection*{Hinweise zur Gestaltung der Interaktion}
\designnotes

% -------------------------------
% WANN
% -------------------------------

\section*{Wann}

\subsection*{Geeigneter Nutzungskontext}
\validcontext

\subsubsection*{Zeit}
\checkbox{\simultaneously} gleichzeitige Nutzung der beteiligten Geräte \\
\checkbox{\sequentially} sequentielle Nutzung der beteiligten Geräte

\subsubsection*{Ort}
\checkbox{\private} privat \\
\checkbox{\semipublic} halb-öffentlich \\
\checkbox{\public} öffentlich \\
\checkbox{\stationary} stationär \\
\checkbox{\onthego} unterwegs 

\subsubsection*{Körperhaltung der Benutzer}
\checkbox{\leanback} Lean-Back \\
\checkbox{\leanforward} Lean-Forward 

\subsubsection*{Teilnehmer}
\checkbox{\single} Einzelnutzer \\
\checkbox{\collaboration} Kollaboration

\subsubsection*{Anordnung zwischen Sender und Empfänger}
\checkbox{\facetoface} Face-To-Face \\
\checkbox{\sidetoside} Side-To-Side

\subsection*{Abzuratender Nutzungskontext}
\notvalidcontext

\subsection*{Geräteklassen}
\devicetabular


% -------------------------------
% WARUM
% -------------------------------

\section*{Warum}
\checkbox{\established} Bewährtes Interaction Pattern \\
\checkbox{\candidate} Interaction Pattern Kandidat: 
\checkbox{\realizable} realisierbar oder
\checkbox{\futuristic} futuristisch

\subsection*{Verwandte Patterns}
\otherpatterns

\subsection*{State of the Art}
\stateoftheart

\subsection*{Checkliste: Entspricht die Interaktion der Definiton einer "Blended Interaction"?}
\checkbox{\designprinciples} Werden die Designprinzipien berücksichtigt?
\begin{itemize}
\item[-] Die Interaktion greift eine Metapher aus der physikalischen Welt auf.
\item[-] Die Interaktion kann in einer Kollaboration ausgeführt werden.
\item[-] Die Interaktion unterstützt einen Workflow/eine Aufgabe.
\item[-] Die Interaktion findet in einer physikalischen Umgebung statt.
\end{itemize} 

\checkbox{\imageschemata} Image Schema/ta liegen zu Grunde.
\begin{itemize}
\writeifexists{\imageSchemaVoid}{Keine Image Schemata vorhanden.}
\writeifexists{\imageSchemaObject}{Object}
\writeifexists{\imageSchemaSubstance}{Substance}
\writeifexists{\imageSchemaCenterPeriphery}{Center-Periphery}
\writeifexists{\imageSchemaContact}{Contact}
\writeifexists{\imageSchemaFrontBack}{Front-Back}
\writeifexists{\imageSchemaLocation}{Location}
\writeifexists{\imageSchemaNearFar}{Near-Far}
\writeifexists{\imageSchemaPath}{Path}
\writeifexists{\imageSchemaSourcePathGoal}{Source-Path-Goal}
\writeifexists{\imageSchemaUpDown}{Up-Down}
\writeifexists{\imageSchemaLeftRight}{Left-Right}
\writeifexists{\imageSchemaContainer}{Container}
\writeifexists{\imageSchemaContent}{Content}
\writeifexists{\imageSchemaInOut}{In-Out}
\writeifexists{\imageSchemaSurface}{Surface}
\writeifexists{\imageSchemaMerging}{Merging}
\writeifexists{\imageSchemaSplitting}{Splitting}
\writeifexists{\imageSchemaMomentum}{Momentum}
\writeifexists{\imageSchemaSelfMotion}{Self-Motion}
\writeifexists{\imageSchemaBigSmall}{Big-Small}
\writeifexists{\imageSchemaFastSlow}{Fast-Slow}
\writeifexists{\imageSchemaPartWhole}{Part-Whole}
\end{itemize}

\checkbox{\realworld} Die real-weltlichen Kenntnisse des Menschen werden berücksichtigt.
\begin{itemize}
\item[-] \checkbox{\realworldNaivePhysic} Naive Physik
\item[-] \checkbox{\realworldBodyAwareness} Body Awareness and Skills
\item[-] \checkbox{\realworldEnvironmentAwareness} Environmental Awareness and Skills
\item[-] \checkbox{\realworldSocialAwareness} Social Awareness and Skills
\end{itemize}

\checkbox{\metaphor} Es ist eine natürliche Interaktion. Metapher/Assoziation: \metaphordesc

% -------------------------------
% TECHNISCHES
% -------------------------------

\ifx\requiredTechnologies\undefined	
\else
	\section*{Technisches}

	\subsection*{Benötigte Technologien}
	\requiredTechnologies

	\ifx\implementation\undefined	
	\else
		\subsection*{Implementierungshinweise}
		\implementation
	\fi
\fi


% -------------------------------
% SONSTIGES
% -------------------------------

\section*{Sonstiges}

\subsection*{Autor/en}
\authors

\subsection*{Versionshistorie}
Erstelldatum: \dateofcreation \\
Letzte Änderung am: \versionhistory


\ifx\comments\undefined	
\else
	\subsection*{Kommentare}
	\comments
\fi

\ifx\questions\undefined	
\else
	\subsection*{Offene Fragen}
\questions
\fi

\listoffigures

\printbibliography

\clearpage

\printglossaries

\end{document}

