\input{header.tex}

\newcommand{\name}{Approach To Extend}

% -------------------------------
% WAS
% -------------------------------

\newcommand{\desc}{Ein Benutzer hat ein mobiles Gerät (das \sendev{}) und möchte eine kollaborative Bildschirmnutzung mit einem zweiten Gerät (dem \recdev{}) über Distanz durchführen.}

\newcommand{\solution}{Der Benutzer nähert (\gls{approach}) sich mit seinem Gerät (dem \sendev{}) einem entfernten Gerät (dem \recdev{}). Nähert er sich dem \recdev{} bis auf einen bestimmten räumlichen Abstand, wird eine Bildschirmkollaboration mit dem \recdev{} initiiert.
}

%\newcommand{\category}{give}
%\newcommand{\category}{take}
%\newcommand{\category}{exchange}
\newcommand{\category}{extend}
%\newcommand{\category}{connect}

% -------------------------------
% WIE
% -------------------------------

\newcommand{\useraction}{Der Anwender nähert sich mit einem portablen \sendev{} dem \recdev{} bis auf eine definierte Distanz.}

\newcommand{\reactionSen}{Besteht keine Verbindung zwischen den Geräten, ist das \sendev{} im Monitoring-Modus, in dem es auf Geräte in der Umgebung wartet, was dem Benutzer visuell durch statisches oder kontinuierliches Feedback sichtbar gemacht werden kann. Unterschreitet der Benutzer den Abstands-Schwellwert, erhält er auf dem \sendev{} ein Feedback über das erkannte \recdev{} und den Beginn des Verbindungsaufbaus (z.B. akustisch oder visuell). Bei \textit{erfolgreichem} oder \textit{gescheiterten} Verbindungsaufbau erhält der Benutzer ein entsprechendes Feedback, das wiederum z.B. aus einem akustischen Signal oder Vibration bestehen kann.}

\newcommand{\reactionRec}{Das Empfangsgerät spielt eine passive Rolle und sollte dann Feedback geben, wenn ein mobiles Gerät erkannt wurde und ein Verbindungsaufbau stattfindet, sodass die Reaktion des \sendev{}s und des \recdev{}s gleichzeitig stattfinden und die Zuordnung der Geräte zueinander über das Feedback sichtbar wird.}

\newcommand{\designnotes}{Die Distanz zwischen \sendev{} und \recdev{} darf nicht zu gering definiert werden, da man sonst das \sendev{} ggf. auf das \recdev{} auflegen müsste.}

% -------------------------------
% WANN
% -------------------------------

\newcommand{\validcontext}{Situationen bei denen man einen Teil seines Bildschirminhalts auf einem anderen Gerät anzeigen möchte (z.B. Präsentationsfolien an einen großen Screen senden).}

\newcommand{\simultaneously}{}
%\newcommand{\sequentially}{}

\newcommand{\online}{}
%\newcommand{\offline}{}

\newcommand{\private}{}
\newcommand{\semipublic}{}
\newcommand{\public}{}
\newcommand{\stationary}{}
%\newcommand{\onthego}{}

\newcommand{\leanback}{}
\newcommand{\leanforward}{}

\newcommand{\single}{}
\newcommand{\collaboration}{}
\newcommand{\facetoface}{}
\newcommand{\sidetoside}{}
%\newcommand{\cornertocorner}{}

%\newcommand{\smalltask}{}
%\newcommand{\repeatedtask}{}
%\newcommand{\locationbased}{}
%\newcommand{\distraction}{}
%\newcommand{\urgent}{}

\newcommand{\notvalidcontext}{Der Nutzungskontext muss bekannt sein, sodass nicht versehentlich vertraulichen Informationen öffentlich gemacht werden.}


\newcommand{\devicetabular}{
\begin{tabular}[H]{|c|c|c|c|c|c|}
\hline
\diagbox{von}{nach}   & Smartwatch & Smartphone & Tablet & Tabletop & Screens \\
\hline
Smartwatch            &            &            &        &          &         \\
\hline
Smartphone            &            &            &   x    &     x     &    x  \\
\hline
Tablet                &            &            &   x    &     x    &     x   \\
\hline
Tabletop              &            &            &        &          &         \\
\hline
Screens               &            &            &        &          &         \\
\hline
\end{tabular} }

% -------------------------------
% WARUM
% -------------------------------

%\newcommand{\established}{}
\newcommand{\candidate}{}
\newcommand{\realizable}{}
%\newcommand{\futuristic}{}

\newcommand{\otherpatterns}{
\begin{itemize}
\item Approach To Connect
\item Approach To Give
\item Approach To Take
\item Leave To Disconnect
\end{itemize}
}

\newcommand{\stateoftheart}{
\begin{enumerate}
\item Theoretische und praktische Ansätze zu \textit{Proxemics}, behandelt auch negative Nutzungskontexte (sog. Dark Patterns) \cite{Marquardt2015}, \cite{Marquardt2012}.
\item Kommunikation verschiedener Personen mit einem Fernseher basierend auf der Nähe \cite{Greenberg2011}.
\item Googles Proximity Beacon API auf Basis von Eddystone \cite{Eddystone2016}
\item Apples iBeacon-Standard zur Erkennung von Proximity \cite{IBeacon2016}
\end{enumerate}
}

\newcommand{\designprinciples}{}

\newcommand{\imageschemata}{}
%\newcommand{\imageSchemaVoid}{}
\newcommand{\imageSchemaObject}{}
%\newcommand{\imageSchemaSubstance}{}
\newcommand{\imageSchemaCenterPeriphery}{}
%\newcommand{\imageSchemaContact}{}
%\newcommand{\imageSchemaFrontBack}{}
\newcommand{\imageSchemaLocation}{}
\newcommand{\imageSchemaNearFar}{}
%\newcommand{\imageSchemaPath}{}
\newcommand{\imageSchemaSourcePathGoal}{}
\newcommand{\imageSchemaScale}{}
%\newcommand{\imageSchemaLeftRight}{}
\newcommand{\imageSchemaContainer}{}
\newcommand{\imageSchemaContent}{}
%\newcommand{\imageSchemaFullEmpty}{}
%\newcommand{\imageSchemaInOut}{}
%\newcommand{\imageSchemaSurface}{}
%\newcommand{\imageSchemaMerging}{}
\newcommand{\imageSchemaSplitting}{}
%\newcommand{\imageSchemaMomentum}{}
%\newcommand{\imageSchemaSelfMotion}{}
\newcommand{\imageSchemaBigSmall}{}
%\newcommand{\imageSchemaFastSlow}{}
\newcommand{\imageSchemaPartWhole}{}

\newcommand{\realworld}{}
\newcommand{\realworldNaivePhysic}{}
\newcommand{\realworldBodyAwareness}{}
\newcommand{\realworldEnvironmentAwareness}{}
\newcommand{\realworldSocialAwareness}{}

\newcommand{\metaphor}{}
\newcommand{\metaphordesc}{Magnet, Bildschirmerweiterung}

% -------------------------------
% TECHNISCHES
% -------------------------------

\newcommand{\requiredTechnologies}{
Um Approach To Extend auf den beteiligten Geräten ausführen zu können, gibt es einige Voraussetzungen und Einschränkungen bezüglich der verfügbaren Technologien auf diesen Geräten. Ein Gerät ist dann für das Approach To Extend Pattern verwendbar, wenn es folgende Eigenschaften aufweist:
\begin{itemize}
\item \textbf{Proximity}: Voraussetzung für die Erkennung der Approach To Extend Geste ist die technische Möglichkeit, Entfernungen zu anderen Geräten messen zu können. Dazu gibt es verschiedene technische Möglichkeiten, wobei zumeist \textit{Bluetooth Low Energy} für weitere und \textit{NFC} für kurze Distanzen weit verbreitet ist.
\item \textbf{Output}: Zur kollaborativen Bildschirmnutzung sind gleichwertige Bildschirme auf beiden Geräten notwendig. Zusätzliches Feedback durch Sound und Vibration ist denkbar, um Anfang und Ende der kollaborativen Bildschirmnutzung zu signalisieren.
\item \textbf{Connectivity}: Ziel der Approach To Extend Geste ist die kollaborative Nutzung
von zwei Geräten, die bereits über ein Ad-Hoc Netzwerk wie Bluetooth oder WiFi Direct miteinander verbunden sind. Alternativ kann auch eine Vermittlungskomponente (Server) genutzt werden, der den Datenaustausch zentral verwaltet.
\end{itemize}

Abbildung \ref{approach_hardware} fasst die benötigten Technologien modellhaft zusammen.

\begin{figure}[h]
\includegraphics[width=\textwidth]{approach_hardware.png}
\caption{Benötigte Technologien für das Approach To Extend Pattern}
\label{approach_hardware}
\end{figure}
}

\newcommand{\implementation}{
\subsubsection*{Ablauf Gestenerkennung}
Bei der Geste Approach To Extend handelt es sich um eine \gls{einfachegeste}, deren Erkennung nur auf einem Gerät durchgeführt werden muss. Der allgemeine Ablauf zerfällt in zwei Teile (s. Abbildung \ref{gesture_detection}):
\begin{itemize}
\item Erkennen der Geste (\textit{Gesture Detection}) und
\item Überprüfen eventueller Bedingungen an die Geste (\textit{Constraint Check}).
\end{itemize}

\begin{figure}[h]
\includegraphics[width=\textwidth]{gesture_detection.png}
\caption{Allgemeiner Ablauf einer Gestenerkennung}
\label{gesture_detection}
\end{figure}
Wurde die Geste erkannt, wird ein entsprechendes \textit{GestureEvent} generiert, das an den \textit{Constraint Check} übergeben wird.

\subsubsection*{Approach Erkennung}
Zum Erkennen der Geste wird \textit{Proximity}-Hardware vorausgesetzt, die \textit{Ranging} ermöglicht, also eine kontinuierliche Suche nach Geräten in der Umgebung. Gefundene Geräte sollten sich mit einer \textit{Service-UUID} o.Ä. identifizieren, anhand derer sie als mögliche Verbindungspartner identifiziert werden können. Daraus ergeben sich folgende Events:
\begin{itemize}
\item \textbf{DEVICE\_FOUND}: Beim Ranging wurde ein entferntes Gerät gefunden.
\item \textbf{MATCH}: Ein gefundenes Gerät besitzt die korrekte Service-UUID.
\item \textbf{NO\_MATCH}: Ein gefundenes Gerät besitzt eine ungültige Service-UUID.
\item \textbf{STOP\_RANGING}: Das Ranging wurde beendet, es werden keine weiteren.
\end{itemize}

Basierend auf diesen Events ergibt sich die Implementierung der Approach-Erkennung entsprechend Abbildung \ref{recognize_approach}. Ein Approach wird dann erkannt, wenn ein beim Ranging gefundenes Gerät die erwartete Service-UUID besitzt. Danach kann entweder nach weiteren Geräten gesucht oder die Suche beendet werden. Besitzt ein gefundenes Gerät keine gültige Service-UUID, wird so lange weiter gesucht, bis entweder ein möglicher Verbindungspartner gefunden wurde oder ein Abbruch des Ranging signalisiert wird. Wurde ein Approach erkannt, wird ein \textit{ProximityEvent} generiert, das für die Überprüfung der \textit{Constraints} im nächsten Schritt benötigt wird.

\begin{figure}[h]
\includegraphics[width=\textwidth]{approach_recognize.png}
\caption{Erkennung der Approach Geste}
\label{recognize_approach}
\end{figure}

\subsubsection*{Approach Constraint Check}
Um einzuschränken, unter welchen Bedingungen eine Annäherung als Approach To Extend gewertet werden soll, wird das im vorigen Schritt generierte \textit{ProximityEvent} mit vorher gesetzten \textit{Constraint} verglichen. Abbildung \ref{check_approach_constraints} zeigt beispielhaft den Ablauf des \textit{Constraint Checks}. Es werden zwei Constraints überprüft:
\begin{itemize}
\item Ein Integer, die sog. Major-Nummer und
\item die Entfernung (\textit{Proximity}) zum erkannten Gerät.
\end{itemize}
Der erste Constraint ist angelehnt an das iBeacon-Protokoll, in dem neben einer Service-UUID (s. voriges Kapitel) zur eindeutigen Identifikation von Geräten eine Major- und Minor-Nummer gesendet werden können. Dadurch kann der Kreis möglicher Verbindungspartner weiter eingeschränkt werden.
Im nächsten Schritt wird die gemessene Entfernung mit einem Vorgabewert verglichen. Die vier Bereiche \textit{immediate}, \textit{near}, \textit{far} und \textit{unknown} sind angelehnt an \cite{Marquardt2015} und können auch durch konkrete Entfernungswerte ersetzt werden.\\

Entsprechend dem Lebenszyklus einer Multiscreen-Applikation ist die Voraussetzung
für Approach To Extend eine bestehende Verbindung zum \recdev{} (vorangegangener Connect) und die Auswahl zu teilender Daten (Select) auf dem \sendev{}. Die Initiative zum Teilen des Bildschirms geht vom \sendev{} aus, statt einer Übertragung diskreter Daten wird ein kontinuierlicher Datenstrom übertragen und vom \recdev{} verarbeitet, bis die Verbindung beendet wird.\\

Weitergehende Informationen zum Applikations-Lebenszyklus und den weiteren
Gestaltungsmöglichkeiten für den \textit{Transfer}, \textit{Feedbacks} etc. finden sich
auf der \developerpage.

\begin{figure}[h]
\includegraphics[width=\textwidth]{approach_check_constraints.png}
\caption{Überprüfung der Constraints für die Approach Geste}
\label{check_approach_constraints}
\end{figure}
}

% -------------------------------
% SONSTIGES
% -------------------------------

\newcommand{\authors}{Horst Schneider, Hochschule Mannheim\\
Dominick Madden, Hochschule Mannheim\\
Valentina Burjan, Hochschule Mannheim}
\newcommand{\versionhistory}{08.05.2017}
\newcommand{\dateofcreation}{15.08.2015}

% template inkludieren --------------

\begin{document}

% ------ fixes the build for all patterns where those new variables haven't been defined yet

%\ifdefined\reactionSen
%\else
%\newcommand{\reactionSen}{tbd.}
%\fi
	
%\ifdefined\reactionRec
%\else
%\newcommand{\reactionRec}{tbd.}
%\fi

%\ifdefined\microinteractionstabular
%\else
%\newcommand{\microinteractionstabular}{tbd.}
%\fi

%\ifdefined\animations
%\else
%\newcommand{\animations}{tbd.}
%\fi

%\ifdefined\requiredTechnologies
%\else
%\newcommand{\requiredTechnologies}{tbd.}
%\fi

%\ifdefined\implementation
%\else
%\newcommand{\implementation}{tbd.}
%\fi

\maketitle

%----------------------------
% CATEGORY ICON
%----------------------------
\begin{textblock}{2}[0,0](8, -3)
\ifthenelse{\equal{\category}{give}}{\newcommand{\icon}{icon_give.png}}{}
\ifthenelse{\equal{\category}{take}}{\newcommand{\icon}{icon_take.png}}{}
\ifthenelse{\equal{\category}{connect}}{\newcommand{\icon}{icon_connect.png}}{}
\ifthenelse{\equal{\category}{extend}}{\newcommand{\icon}{icon_extend.png}}{}
\ifthenelse{\equal{\category}{exchange}}{\newcommand{\icon}{icon_exchange.png}}{}	
\includegraphics[scale=0.8]{\icon}
\end{textblock}

% -------------------------------
% WAS
% -------------------------------
\section*{Was}

\subsection*{Problem}
\desc

\subsection*{Lösung}
\solution

\subsection*{Grafische Darstellung}
\begin{figure}[H]
\IfFileExists{\jobname_graphical_description.png}{\includegraphics[width=\textwidth]{\grafischedarstellung}}{}
\end{figure}

\subsection*{Kategorie}
\ifthenelse{\equal{\category}{give}}{$\boxtimes$}{$\Box$} Give   |   
\ifthenelse{\equal{\category}{take}}{$\boxtimes$}{$\Box$} Take   |   
\ifthenelse{\equal{\category}{exchange}}{$\boxtimes$}{$\Box$} Exchange   |   
\ifthenelse{\equal{\category}{extend}}{$\boxtimes$}{$\Box$} Extend   |   
\ifthenelse{\equal{\category}{connect}}{$\boxtimes$}{$\Box$} Connect

% -------------------------------
% WIE
% -------------------------------
\newpage
\section*{Wie}

%%%define swapReactions if the receiver is the primary actor%%%
%%%define usersEqual if there is no sender and receiver%%%
\ifx\reactionSen\undefined
\else
	\ifx\swapReactions\undefined
		\ifx\usersEqual\undefined	
			\subsection*{Aktion des \sen s}
			\useraction

			\subsection*{Reaktionen des \sendev s}
			\reactionSen

			\subsection*{Reaktionen des \recdev s}
			\reactionRec
		\else
			\subsection*{Aktion der Benutzer}
			\useraction

			\subsection*{Reaktionen der Geräte}
			\reactionSen
		\fi
	\else
		\subsection*{Aktion des \rec s}
		\useraction
	
		\subsection*{Reaktionen des \recdev s}
		\reactionRec
	
		\subsection*{Reaktionen des \sendev s}
		\reactionSen
	\fi
\fi	

\ifx\microinteractionstabular\undefined
\else
	\subsection*{Übersicht über die Atomaren Interaktionen}
	\microinteractionstabular
\fi

\ifx\animations\undefined
\else
	\subsubsection*{Animationen}
	\animations
\fi

\subsection*{Hinweise zur Gestaltung der Interaktion}
\designnotes

% -------------------------------
% WANN
% -------------------------------

\section*{Wann}

\subsection*{Geeigneter Nutzungskontext}
\validcontext

\subsubsection*{Zeit}
\checkbox{\simultaneously} gleichzeitige Nutzung der beteiligten Geräte \\
\checkbox{\sequentially} sequentielle Nutzung der beteiligten Geräte

\subsubsection*{Ort}
\checkbox{\private} privat \\
\checkbox{\semipublic} halb-öffentlich \\
\checkbox{\public} öffentlich \\
\checkbox{\stationary} stationär \\
\checkbox{\onthego} unterwegs 

\subsubsection*{Körperhaltung der Benutzer}
\checkbox{\leanback} Lean-Back \\
\checkbox{\leanforward} Lean-Forward 

\subsubsection*{Teilnehmer}
\checkbox{\single} Einzelnutzer \\
\checkbox{\collaboration} Kollaboration

\subsubsection*{Anordnung zwischen Sender und Empfänger}
\checkbox{\facetoface} Face-To-Face \\
\checkbox{\sidetoside} Side-To-Side

\subsection*{Abzuratender Nutzungskontext}
\notvalidcontext

\subsection*{Geräteklassen}
\devicetabular


% -------------------------------
% WARUM
% -------------------------------

\section*{Warum}
\checkbox{\established} Bewährtes Interaction Pattern \\
\checkbox{\candidate} Interaction Pattern Kandidat: 
\checkbox{\realizable} realisierbar oder
\checkbox{\futuristic} futuristisch

\subsection*{Verwandte Patterns}
\otherpatterns

\subsection*{State of the Art}
\stateoftheart

\subsection*{Checkliste: Entspricht die Interaktion der Definiton einer "Blended Interaction"?}
\checkbox{\designprinciples} Werden die Designprinzipien berücksichtigt?
\begin{itemize}
\item[-] Die Interaktion greift eine Metapher aus der physikalischen Welt auf.
\item[-] Die Interaktion kann in einer Kollaboration ausgeführt werden.
\item[-] Die Interaktion unterstützt einen Workflow/eine Aufgabe.
\item[-] Die Interaktion findet in einer physikalischen Umgebung statt.
\end{itemize} 

\checkbox{\imageschemata} Image Schema/ta liegen zu Grunde.
\begin{itemize}
\writeifexists{\imageSchemaVoid}{Keine Image Schemata vorhanden.}
\writeifexists{\imageSchemaObject}{Object}
\writeifexists{\imageSchemaSubstance}{Substance}
\writeifexists{\imageSchemaCenterPeriphery}{Center-Periphery}
\writeifexists{\imageSchemaContact}{Contact}
\writeifexists{\imageSchemaFrontBack}{Front-Back}
\writeifexists{\imageSchemaLocation}{Location}
\writeifexists{\imageSchemaNearFar}{Near-Far}
\writeifexists{\imageSchemaPath}{Path}
\writeifexists{\imageSchemaSourcePathGoal}{Source-Path-Goal}
\writeifexists{\imageSchemaUpDown}{Up-Down}
\writeifexists{\imageSchemaLeftRight}{Left-Right}
\writeifexists{\imageSchemaContainer}{Container}
\writeifexists{\imageSchemaContent}{Content}
\writeifexists{\imageSchemaInOut}{In-Out}
\writeifexists{\imageSchemaSurface}{Surface}
\writeifexists{\imageSchemaMerging}{Merging}
\writeifexists{\imageSchemaSplitting}{Splitting}
\writeifexists{\imageSchemaMomentum}{Momentum}
\writeifexists{\imageSchemaSelfMotion}{Self-Motion}
\writeifexists{\imageSchemaBigSmall}{Big-Small}
\writeifexists{\imageSchemaFastSlow}{Fast-Slow}
\writeifexists{\imageSchemaPartWhole}{Part-Whole}
\end{itemize}

\checkbox{\realworld} Die real-weltlichen Kenntnisse des Menschen werden berücksichtigt.
\begin{itemize}
\item[-] \checkbox{\realworldNaivePhysic} Naive Physik
\item[-] \checkbox{\realworldBodyAwareness} Body Awareness and Skills
\item[-] \checkbox{\realworldEnvironmentAwareness} Environmental Awareness and Skills
\item[-] \checkbox{\realworldSocialAwareness} Social Awareness and Skills
\end{itemize}

\checkbox{\metaphor} Es ist eine natürliche Interaktion. Metapher/Assoziation: \metaphordesc

% -------------------------------
% TECHNISCHES
% -------------------------------

\ifx\requiredTechnologies\undefined	
\else
	\section*{Technisches}

	\subsection*{Benötigte Technologien}
	\requiredTechnologies

	\ifx\implementation\undefined	
	\else
		\subsection*{Implementierungshinweise}
		\implementation
	\fi
\fi


% -------------------------------
% SONSTIGES
% -------------------------------

\section*{Sonstiges}

\subsection*{Autor/en}
\authors

\subsection*{Versionshistorie}
Erstelldatum: \dateofcreation \\
Letzte Änderung am: \versionhistory


\ifx\comments\undefined	
\else
	\subsection*{Kommentare}
	\comments
\fi

\ifx\questions\undefined	
\else
	\subsection*{Offene Fragen}
\questions
\fi

\listoffigures

\printbibliography

\clearpage

\printglossaries

\end{document}

