\input{header.tex}

\newcommand{\name}{Shake To Connect}

% -------------------------------
% WAS
% -------------------------------

\newcommand{\desc}{Eine Verbindung soll zwischen zwei Geräten hergestellt werden.}

\newcommand{\solution}{Ein Anwender hat zwei handliche Geräte fest in einer Hand. Durch eine Schüttel-Bewegung wird die Verbindung zwischen den zwei Geräten hergestellt.}

%\newcommand{\category}{give}
%\newcommand{\category}{take}
%\newcommand{\category}{exchange}
%\newcommand{\category}{extend}
\newcommand{\category}{connect}

% -------------------------------
% WIE
% -------------------------------

\newcommand{\useraction}{Der Anwender hält zwei Geräte in einer Hand fest umschlossen. Anschließend schüttelt der Anwender die Geräte solange bis sie eine Verbindung zueinander aufgebaut haben.}

\newcommand{\reaction}{Sobald der Anwender die Schüttel-Bewegung ausübt, analysieren die Geräte bzw. die Anwendungen auf den Geräten die Bewegung. Ist die Bewegung auf beiden Geräten identisch werden die zwei Geräte miteinander verbunden. Bei erfolgreicher Verbindung wird der Anwender (z.B. akustisch oder sensitiv) informiert.}

%\newcommand{\reactionSuccessVisual}{}
\newcommand{\reactionSuccessAcustic}{}
\newcommand{\reactionSuccessSensitive}{}

\newcommand{\reactionFailureConnection}{}
\newcommand{\reactionFailureConnectionDesc}{sollte der Anwender informiert werden (akustisch), zum Beispiel wenn die Schüttel-Bewegung länger als eine definierte Zeit (bspw. 3 Sekunden) erfolglos durchgeführt wird.}

\newcommand{\reactionFailureNoDevice}{}
\newcommand{\reactionFailureNoDeviceDesc}{wurde von den Geräten eine unterschiedliche Bewegung festgestellt. Der Anwender sollte die Interaktion erneut ausführen.}

%\newcommand{\reactionFailureCompatibility}{}
\newcommand{\reactionFailureCompatibilityDesc}{...}

\newcommand{\designnotes}{
\begin{itemize}
\item Die Geräte müssen handlich sein (z.B. Smartphones), um die Interaktion korrekt und erfolgreich durchzuführen. Größere Geräte könnten ggf. beschädigt werden.
\item Der Anwender muss die Geräte in einer Hand halten. Die Interaktionsausführung mit zwei Händen kann nicht erfolgreich ausgeführt werden, da die Bewegungen mit zwei Händen nicht synchron verlaufen können.
\end{itemize}}
% -------------------------------
% WANN
% -------------------------------

\newcommand{\validcontext}{...}

\newcommand{\simultaneously}{}
%\newcommand{\sequentially}{}

\newcommand{\online}{}
\newcommand{\offline}{}

\newcommand{\private}{}
\newcommand{\semipublic}{}
\newcommand{\public}{}
\newcommand{\stationary}{}
\newcommand{\onthego}{}

\newcommand{\leanback}{}
\newcommand{\leanforward}{}

\newcommand{\single}{}
%\newcommand{\collaboration}{}
%\newcommand{\facetoface}{}
%\newcommand{\sidetoside}{}
%\newcommand{\cornertocorner}{}

%\newcommand{\smalltask}{}
%\newcommand{\repeatedtask}{}
%\newcommand{\locationbased}{}
%\newcommand{\distraction}{}
%\newcommand{\urgent}{} 

\newcommand{\notvalidcontext}{...}


\newcommand{\devicetabular}{
\begin{tabular}[H]{|c|c|c|c|c|c|}
\hline 
\diagbox{von}{nach}   & Smartwatch & Smartphone & Tablet & Tabletop & Screens \\ 
\hline 
Smartwatch            &            &            &        &          &         \\ 
\hline 
Smartphone            &            &     x      &        &          &         \\ 
\hline 
Tablet                &            &            &        &          &         \\ 
\hline 
Tabletop              &            &            &        &          &         \\ 
\hline
Screens               &            &            &        &          &         \\ 
\hline 
\end{tabular} }

% -------------------------------
% WARUM
% -------------------------------

%\newcommand{\established}{}
\newcommand{\candidate}{}
\newcommand{\realizable}{}
%\newcommand{\futuristic}{}

\newcommand{\otherpatterns}{---}

\newcommand{\stateoftheart}{
\begin{enumerate}
\item
\end{enumerate}
}


\newcommand{\designprinciples}{}

%\newcommand{\imageschemata}{}
\newcommand{\imageSchemaVoid}{}
%\newcommand{\imageSchemaObject}{}
%\newcommand{\imageSchemaSubstance}{}
%\newcommand{\imageSchemaCenterPeriphery}{}
%\newcommand{\imageSchemaContact}{}
%\newcommand{\imageSchemaFrontBack}{}
%\newcommand{\imageSchemaLocation}{}
%\newcommand{\imageSchemaNearFar}{}
%\newcommand{\imageSchemaPath}{}
%\newcommand{\imageSchemaSourcePathGoal}{}
%\newcommand{\imageSchemaScale}{}
%\newcommand{\imageSchemaLeftRight}{}
%\newcommand{\imageSchemaContainer}{}
%\newcommand{\imageSchemaContent}{}
%\newcommand{\imageSchemaFullEmpty}{}
%\newcommand{\imageSchemaInOut}{}
%\newcommand{\imageSchemaSurface}{}
%\newcommand{\imageSchemaMerging}{}
%\newcommand{\imageSchemaSplitting}{}
%\newcommand{\imageSchemaMomentum}{}
%\newcommand{\imageSchemaSelfMotion}{}
%\newcommand{\imageSchemaBigSmall}{}
%\newcommand{\imageSchemaFastSlow}{}
%\newcommand{\imageSchemaPartWhole}{}

\newcommand{\realworld}{}
\newcommand{\realworldNaivePhysic}{}
\newcommand{\realworldBodyAwareness}{}
\newcommand{\realworldEnvironmentAwareness}{}
\newcommand{\realworldSocialAwareness}{}

\newcommand{\metaphor}{}
\newcommand{\metaphordesc}{etwas schütteln}

% -------------------------------
% TECHNISCHES
% -------------------------------

\newcommand{\technologyObjectIntimate}{}
%\newcommand{\technologyObjectPersonal}{}
%\newcommand{\technologyObjectSocial}{}
%\newcommand{\technologyObjectPublic}{}

\newcommand{\technologyObjectDesc}{...}

%\newcommand{\technologyCommunicationServer}{}
%\newcommand{\technologyCommunicationAdhoc}{}

\newcommand{\technologyCommunicationDesc}{...}

%\newcommand{\technologyOrientationAccelerometer}{}
%\newcommand{\technologyOrientationGPS}{}
%\newcommand{\technologyOrientationGyroskop}{}
%\newcommand{\technologyOrientationAnnaeherung}{}
%\newcommand{\technologyOrientationHoehe}{}
%\newcommand{\technologyOrientationBeacons}{}
%\newcommand{\technologyOrientationOther}{}

\newcommand{\technologyOrientationDesc}{...}

\newcommand{\prototype}{...}


% -------------------------------
% SONSTIGES
% -------------------------------

\newcommand{\authors}{...}
\newcommand{\literature}{...}
\newcommand{\figures}{...}
\newcommand{\versionhistory}{...}
\newcommand{\dateofcreation}{...}
\newcommand{\comments}{...}
\newcommand{\questions}{...}


% template inkludieren --------------

\begin{document}

% ------ fixes the build for all patterns where those new variables haven't been defined yet

%\ifdefined\reactionSen
%\else
%\newcommand{\reactionSen}{tbd.}
%\fi
	
%\ifdefined\reactionRec
%\else
%\newcommand{\reactionRec}{tbd.}
%\fi

%\ifdefined\microinteractionstabular
%\else
%\newcommand{\microinteractionstabular}{tbd.}
%\fi

%\ifdefined\animations
%\else
%\newcommand{\animations}{tbd.}
%\fi

%\ifdefined\requiredTechnologies
%\else
%\newcommand{\requiredTechnologies}{tbd.}
%\fi

%\ifdefined\implementation
%\else
%\newcommand{\implementation}{tbd.}
%\fi

\maketitle

%----------------------------
% CATEGORY ICON
%----------------------------
\begin{textblock}{2}[0,0](8, -3)
\ifthenelse{\equal{\category}{give}}{\newcommand{\icon}{icon_give.png}}{}
\ifthenelse{\equal{\category}{take}}{\newcommand{\icon}{icon_take.png}}{}
\ifthenelse{\equal{\category}{connect}}{\newcommand{\icon}{icon_connect.png}}{}
\ifthenelse{\equal{\category}{extend}}{\newcommand{\icon}{icon_extend.png}}{}
\ifthenelse{\equal{\category}{exchange}}{\newcommand{\icon}{icon_exchange.png}}{}	
\includegraphics[scale=0.8]{\icon}
\end{textblock}

% -------------------------------
% WAS
% -------------------------------
\section*{Was}

\subsection*{Problem}
\desc

\subsection*{Lösung}
\solution

\subsection*{Grafische Darstellung}
\begin{figure}[H]
\IfFileExists{\jobname_graphical_description.png}{\includegraphics[width=\textwidth]{\grafischedarstellung}}{}
\end{figure}

\subsection*{Kategorie}
\ifthenelse{\equal{\category}{give}}{$\boxtimes$}{$\Box$} Give   |   
\ifthenelse{\equal{\category}{take}}{$\boxtimes$}{$\Box$} Take   |   
\ifthenelse{\equal{\category}{exchange}}{$\boxtimes$}{$\Box$} Exchange   |   
\ifthenelse{\equal{\category}{extend}}{$\boxtimes$}{$\Box$} Extend   |   
\ifthenelse{\equal{\category}{connect}}{$\boxtimes$}{$\Box$} Connect

% -------------------------------
% WIE
% -------------------------------
\newpage
\section*{Wie}

%%%define swapReactions if the receiver is the primary actor%%%
%%%define usersEqual if there is no sender and receiver%%%
\ifx\reactionSen\undefined
\else
	\ifx\swapReactions\undefined
		\ifx\usersEqual\undefined	
			\subsection*{Aktion des \sen s}
			\useraction

			\subsection*{Reaktionen des \sendev s}
			\reactionSen

			\subsection*{Reaktionen des \recdev s}
			\reactionRec
		\else
			\subsection*{Aktion der Benutzer}
			\useraction

			\subsection*{Reaktionen der Geräte}
			\reactionSen
		\fi
	\else
		\subsection*{Aktion des \rec s}
		\useraction
	
		\subsection*{Reaktionen des \recdev s}
		\reactionRec
	
		\subsection*{Reaktionen des \sendev s}
		\reactionSen
	\fi
\fi	

\ifx\microinteractionstabular\undefined
\else
	\subsection*{Übersicht über die Atomaren Interaktionen}
	\microinteractionstabular
\fi

\ifx\animations\undefined
\else
	\subsubsection*{Animationen}
	\animations
\fi

\subsection*{Hinweise zur Gestaltung der Interaktion}
\designnotes

% -------------------------------
% WANN
% -------------------------------

\section*{Wann}

\subsection*{Geeigneter Nutzungskontext}
\validcontext

\subsubsection*{Zeit}
\checkbox{\simultaneously} gleichzeitige Nutzung der beteiligten Geräte \\
\checkbox{\sequentially} sequentielle Nutzung der beteiligten Geräte

\subsubsection*{Ort}
\checkbox{\private} privat \\
\checkbox{\semipublic} halb-öffentlich \\
\checkbox{\public} öffentlich \\
\checkbox{\stationary} stationär \\
\checkbox{\onthego} unterwegs 

\subsubsection*{Körperhaltung der Benutzer}
\checkbox{\leanback} Lean-Back \\
\checkbox{\leanforward} Lean-Forward 

\subsubsection*{Teilnehmer}
\checkbox{\single} Einzelnutzer \\
\checkbox{\collaboration} Kollaboration

\subsubsection*{Anordnung zwischen Sender und Empfänger}
\checkbox{\facetoface} Face-To-Face \\
\checkbox{\sidetoside} Side-To-Side

\subsection*{Abzuratender Nutzungskontext}
\notvalidcontext

\subsection*{Geräteklassen}
\devicetabular


% -------------------------------
% WARUM
% -------------------------------

\section*{Warum}
\checkbox{\established} Bewährtes Interaction Pattern \\
\checkbox{\candidate} Interaction Pattern Kandidat: 
\checkbox{\realizable} realisierbar oder
\checkbox{\futuristic} futuristisch

\subsection*{Verwandte Patterns}
\otherpatterns

\subsection*{State of the Art}
\stateoftheart

\subsection*{Checkliste: Entspricht die Interaktion der Definiton einer "Blended Interaction"?}
\checkbox{\designprinciples} Werden die Designprinzipien berücksichtigt?
\begin{itemize}
\item[-] Die Interaktion greift eine Metapher aus der physikalischen Welt auf.
\item[-] Die Interaktion kann in einer Kollaboration ausgeführt werden.
\item[-] Die Interaktion unterstützt einen Workflow/eine Aufgabe.
\item[-] Die Interaktion findet in einer physikalischen Umgebung statt.
\end{itemize} 

\checkbox{\imageschemata} Image Schema/ta liegen zu Grunde.
\begin{itemize}
\writeifexists{\imageSchemaVoid}{Keine Image Schemata vorhanden.}
\writeifexists{\imageSchemaObject}{Object}
\writeifexists{\imageSchemaSubstance}{Substance}
\writeifexists{\imageSchemaCenterPeriphery}{Center-Periphery}
\writeifexists{\imageSchemaContact}{Contact}
\writeifexists{\imageSchemaFrontBack}{Front-Back}
\writeifexists{\imageSchemaLocation}{Location}
\writeifexists{\imageSchemaNearFar}{Near-Far}
\writeifexists{\imageSchemaPath}{Path}
\writeifexists{\imageSchemaSourcePathGoal}{Source-Path-Goal}
\writeifexists{\imageSchemaUpDown}{Up-Down}
\writeifexists{\imageSchemaLeftRight}{Left-Right}
\writeifexists{\imageSchemaContainer}{Container}
\writeifexists{\imageSchemaContent}{Content}
\writeifexists{\imageSchemaInOut}{In-Out}
\writeifexists{\imageSchemaSurface}{Surface}
\writeifexists{\imageSchemaMerging}{Merging}
\writeifexists{\imageSchemaSplitting}{Splitting}
\writeifexists{\imageSchemaMomentum}{Momentum}
\writeifexists{\imageSchemaSelfMotion}{Self-Motion}
\writeifexists{\imageSchemaBigSmall}{Big-Small}
\writeifexists{\imageSchemaFastSlow}{Fast-Slow}
\writeifexists{\imageSchemaPartWhole}{Part-Whole}
\end{itemize}

\checkbox{\realworld} Die real-weltlichen Kenntnisse des Menschen werden berücksichtigt.
\begin{itemize}
\item[-] \checkbox{\realworldNaivePhysic} Naive Physik
\item[-] \checkbox{\realworldBodyAwareness} Body Awareness and Skills
\item[-] \checkbox{\realworldEnvironmentAwareness} Environmental Awareness and Skills
\item[-] \checkbox{\realworldSocialAwareness} Social Awareness and Skills
\end{itemize}

\checkbox{\metaphor} Es ist eine natürliche Interaktion. Metapher/Assoziation: \metaphordesc

% -------------------------------
% TECHNISCHES
% -------------------------------

\ifx\requiredTechnologies\undefined	
\else
	\section*{Technisches}

	\subsection*{Benötigte Technologien}
	\requiredTechnologies

	\ifx\implementation\undefined	
	\else
		\subsection*{Implementierungshinweise}
		\implementation
	\fi
\fi


% -------------------------------
% SONSTIGES
% -------------------------------

\section*{Sonstiges}

\subsection*{Autor/en}
\authors

\subsection*{Versionshistorie}
Erstelldatum: \dateofcreation \\
Letzte Änderung am: \versionhistory


\ifx\comments\undefined	
\else
	\subsection*{Kommentare}
	\comments
\fi

\ifx\questions\undefined	
\else
	\subsection*{Offene Fragen}
\questions
\fi

\listoffigures

\printbibliography

\clearpage

\printglossaries

\end{document}
