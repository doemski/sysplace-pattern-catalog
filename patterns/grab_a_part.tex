\input{header.tex}

\newcommand{\name}{Grab A Part}

% -------------------------------
% WAS
% -------------------------------

\newcommand{\desc}{Ein Teil eines Bildschirmes von einem Gerät soll mit einem anderen Gerät ausgewählt und aufgesaugt werden, sodass der gewählte Ausschnitt auf dem anderen Gerät verfügbar ist.}

\newcommand{\solution}{Durch Auflegen eines \recdev{}s auf ein \sendev{} wird ein ausgewählter Bildausschnitt vom \sendev{} auf das \recdev{} übertragen. Nach dem Wiederwegnehmen ist der Ausschnitt auch auf dem \recdev{} verfügbar.}

%\newcommand{\category}{give}
\newcommand{\category}{take}
%\newcommand{\category}{exchange}
%\newcommand{\category}{extend}
%\newcommand{\category}{connect}

% -------------------------------
% WIE
% -------------------------------

\newcommand{\useraction}{Der Anwender legt ein \recdev{} auf ein \sendev{}.
Anschließend nimmt der Anwender das \recdev{} wieder weg.}

\newcommand{\reaction}{Das ausgewählte Datenobjekt auf dem \sendev{} wird auf dem Bildschirm des \recdev{} dargestellt.}

\newcommand{\reactionSuccessVisual}{}
%\newcommand{\reactionSuccessAcustic}{}
%\newcommand{\reactionSuccessSensitive}{}

%\newcommand{\reactionFailureConnection}{}
\newcommand{\reactionFailureConnectionDesc}{ist das "Mitnehmen" von den Daten nicht möglich. Der Nutzer muss vor Interaktionsauführung darauf hingewiesen werden (z.B. visuell), dass noch keine Verbindung zwischen den Geräten hergestellt wurde. Der Nutzer muss ggf. prüfen, ob die Internetverbindung und/oder andere Schnittstellen aktiv sind.}

%\newcommand{\reactionFailureNoDevice}{}
\newcommand{\reactionFailureNoDeviceDesc}{ist das "Mitnehmen" von den Daten nicht möglich. Der Nutzer muss ggf. die Interaktion erneut ausführen.}

%\newcommand{\reactionFailureCompatibility}{}
\newcommand{\reactionFailureCompatibilityDesc}{ist die Interaktion zwischen den gewählten Geräten nicht erfolgreich ausführbar.}

\newcommand{\designnotes}{
\begin{itemize}
\item[-] Die Geräte müssen über einen Bildschirm verfügen, um die Daten darzustellen.
\item[-] Die Geräte müssen über definierte Schnittstellen miteinander kommunizieren, um das Datenobjekt zu senden und zu empfangen.
\item[-] Für eine intuitive Durchführung der Interaktion muss das Gerät, das die Daten bereitstellt, größer sein als das Gerät, das aufgelegt wird und das Datenobjekt mitnimmt/aufsaugt.
\end{itemize}}

% -------------------------------
% WANN
% -------------------------------

\newcommand{\validcontext}{Informationsbeschaffung, z.B. Stadtkarte}

%\newcommand{\simultaneously}{}
\newcommand{\sequentially}{}

\newcommand{\online}{}
%\newcommand{\offline}{}

%\newcommand{\private}{}
\newcommand{\semipublic}{}
\newcommand{\public}{}
\newcommand{\stationary}{}
%\newcommand{\onthego}{}

%\newcommand{\leanback}{}
\newcommand{\leanforward}{}

\newcommand{\single}{}
\newcommand{\collaboration}{}
%\newcommand{\facetoface}{}
\newcommand{\sidetoside}{}
\newcommand{\cornertocorner}{}

%\newcommand{\smalltask}{}
%\newcommand{\repeatedtask}{}
\newcommand{\locationbased}{}
%\newcommand{\distraction}{}
%\newcommand{\urgent}{}

\newcommand{\notvalidcontext}{Wenn die Fläche des ausgewählten Ausschnitts auf dem \sendev{} größer ist als das Display des \recdev{}s.}


\newcommand{\devicetabular}{
\begin{tabular}[H]{|c|c|c|c|c|c|}
\hline
\diagbox{von}{nach}   & Smartwatch & Smartphone & Tablet & Tabletop & Screens \\
\hline
Smartwatch            &            &            &        &          &         \\
\hline
Smartphone            &            &            &   x    &     x    &     x   \\
\hline
Tablet                &            &            &        &     x    &     x   \\
\hline
Tabletop              &            &            &        &          &         \\
\hline
Screens               &            &            &        &          &         \\
\hline
\end{tabular} }

% -------------------------------
% WARUM
% -------------------------------

%\newcommand{\established}{}
\newcommand{\candidate}{}
\newcommand{\realizable}{}
%\newcommand{\futuristic}{}

\newcommand{\otherpatterns}{Grab An Object - analoge Interaktion, aber hier wird ein zuvor definiertes Objekt (aus einer endlichen Menge) durch das Auflegen des \recdev{}s ausgewählt.}

\newcommand{\stateoftheart}{...
\begin{enumerate}
\item \url{https://www.youtube.com/watch?v=6Cf7IL_eZ38} \\
von Minute 2:50 bis 2:53 \\
Hier wird das Gerät nicht direkt aufgelegt auf das andere Gerät, sondern wie beim Abfotografieren lediglich davorgehalten und die Interaktion „Grab A Part“ durchgeführt.
\end{enumerate}}


\newcommand{\designprinciples}{}

\newcommand{\imageschemata}{}
\newcommand{\imageSchemaContainer}{}
\newcommand{\imageSchemaInOut}{}
%\newcommand{\imageSchemaPath}{}
\newcommand{\imageSchemaSourcePathGoal}{}
%\newcommand{\imageSchemaUpDown}{}
%\newcommand{\imageSchemaLeftRight}{}
%\newcommand{\imageSchemaNearFar}{}
\newcommand{\imageSchemaPartWhole}{}

\newcommand{\realworld}{}
\newcommand{\realworldNaivePhysic}{}
\newcommand{\realworldBodyAwareness}{}
\newcommand{\realworldEnvironmentAwareness}{}
\newcommand{\realworldSocialAwareness}{}

\newcommand{\metaphor}{}
\newcommand{\metaphordesc}{Ein Medium hinlegen, wie ein Schwamm, das den Inhalt von dem anderen Medium aufsaugt. Ausstanzen. Ausschneiden. Screenshot. Nach etwas greifen/aufheben.}

% -------------------------------
% TECHNISCHES
% -------------------------------

\newcommand{\technologyObjectIntimate}{}
\newcommand{\technologyObjectPersonal}{}
\newcommand{\technologyObjectSocial}{}
%\newcommand{\technologyObjectPublic}{}

\newcommand{\technologyObjectDesc}{Um das aufliegende Smartphone auf dem Multi-Touch-Tisch zu erkennen benötigt es eine Technologie (wie Tags), wenn der Multi-Touch-Tisch keine Schnittstelle wie Bluetooth oder NFC anbietet. Sofern der Multi-Touch-Tisch und das Smartphone aber über eine drahtlose Schnittstelle wie Bluetooth kommunizieren können ist die Objekterkennung auch in einer entfernteren Distanz möglich. \\
In weiteren Prototypen (\url{http://hci.uni-konstanz.de/downloads/HuddleLamp_ITS2014.pdf}) ist auch die Kamera als Objekterkennung und zur Lokalisation verwendet worden.}

\newcommand{\technologyCommunicationServer}{}
\newcommand{\technologyCommunicationAdhoc}{}

\newcommand{\technologyCommunicationDesc}{Damit der Multi-Touch-Tisch und das Smartphone kommunizieren (Daten austauschen) können, kann ein Server zwischengeschaltet werden. \\
Wenn technisch eine direkte Kommunikation zwischen Multi-Touch-Tisch und Smartphone (z.B. über Bluetooth) möglich ist, kann auch ein Ad-hoc-Netz gewählt werden.}

%\newcommand{\technologyOrientationAccelerometer}{}
%\newcommand{\technologyOrientationGPS}{}
%\newcommand{\technologyOrientationGyroskop}{}
%\newcommand{\technologyOrientationAnnaeherung}{}
%\newcommand{\technologyOrientationHoehe}{}
\newcommand{\technologyOrientationBeacons}{}
%\newcommand{\technologyOrientationOther}{}

\newcommand{\technologyOrientationDesc}{Um zu bestimmen, ob das Smartphone auf dem Multi-Touch-Tisch aufliegt bzw. in seiner Nähe ist, dienen (wie in [1.] verwendet) Beacons.}

\newcommand{\prototype}{In [1.] ist die prototypische Implementierung der Interaktion unter dem Namen „grab the screen“ beschrieben. Die Beschreibung gilt für einen Multi-Touch-Tisch in Kombination mit einem Android Smartphone. Über einen Server erfolgt die Datenkommunikation und die Lokalisation des Tisches wird durch ein Beacon gestützt.}


% -------------------------------
% SONSTIGES
% -------------------------------

\newcommand{\authors}{Valentina Burjan, Hochschule Mannheim}
\newcommand{\literature}{\begin{enumerate}
\item Burjan, Valentina (2015). Prototypische Implementierung des Blended Interaction Patterns „grab the screen“. Hochschule Mannheim.
\end{enumerate}}
\newcommand{\figures}{...}
\newcommand{\versionhistory}{10.05.2017}
\newcommand{\dateofcreation}{07.05.2016}


% template inkludieren --------------

\begin{document}

% ------ fixes the build for all patterns where those new variables haven't been defined yet

%\ifdefined\reactionSen
%\else
%\newcommand{\reactionSen}{tbd.}
%\fi
	
%\ifdefined\reactionRec
%\else
%\newcommand{\reactionRec}{tbd.}
%\fi

%\ifdefined\microinteractionstabular
%\else
%\newcommand{\microinteractionstabular}{tbd.}
%\fi

%\ifdefined\animations
%\else
%\newcommand{\animations}{tbd.}
%\fi

%\ifdefined\requiredTechnologies
%\else
%\newcommand{\requiredTechnologies}{tbd.}
%\fi

%\ifdefined\implementation
%\else
%\newcommand{\implementation}{tbd.}
%\fi

\maketitle

%----------------------------
% CATEGORY ICON
%----------------------------
\begin{textblock}{2}[0,0](8, -3)
\ifthenelse{\equal{\category}{give}}{\newcommand{\icon}{icon_give.png}}{}
\ifthenelse{\equal{\category}{take}}{\newcommand{\icon}{icon_take.png}}{}
\ifthenelse{\equal{\category}{connect}}{\newcommand{\icon}{icon_connect.png}}{}
\ifthenelse{\equal{\category}{extend}}{\newcommand{\icon}{icon_extend.png}}{}
\ifthenelse{\equal{\category}{exchange}}{\newcommand{\icon}{icon_exchange.png}}{}	
\includegraphics[scale=0.8]{\icon}
\end{textblock}

% -------------------------------
% WAS
% -------------------------------
\section*{Was}

\subsection*{Problem}
\desc

\subsection*{Lösung}
\solution

\subsection*{Grafische Darstellung}
\begin{figure}[H]
\IfFileExists{\jobname_graphical_description.png}{\includegraphics[width=\textwidth]{\grafischedarstellung}}{}
\end{figure}

\subsection*{Kategorie}
\ifthenelse{\equal{\category}{give}}{$\boxtimes$}{$\Box$} Give   |   
\ifthenelse{\equal{\category}{take}}{$\boxtimes$}{$\Box$} Take   |   
\ifthenelse{\equal{\category}{exchange}}{$\boxtimes$}{$\Box$} Exchange   |   
\ifthenelse{\equal{\category}{extend}}{$\boxtimes$}{$\Box$} Extend   |   
\ifthenelse{\equal{\category}{connect}}{$\boxtimes$}{$\Box$} Connect

% -------------------------------
% WIE
% -------------------------------
\newpage
\section*{Wie}

%%%define swapReactions if the receiver is the primary actor%%%
%%%define usersEqual if there is no sender and receiver%%%
\ifx\reactionSen\undefined
\else
	\ifx\swapReactions\undefined
		\ifx\usersEqual\undefined	
			\subsection*{Aktion des \sen s}
			\useraction

			\subsection*{Reaktionen des \sendev s}
			\reactionSen

			\subsection*{Reaktionen des \recdev s}
			\reactionRec
		\else
			\subsection*{Aktion der Benutzer}
			\useraction

			\subsection*{Reaktionen der Geräte}
			\reactionSen
		\fi
	\else
		\subsection*{Aktion des \rec s}
		\useraction
	
		\subsection*{Reaktionen des \recdev s}
		\reactionRec
	
		\subsection*{Reaktionen des \sendev s}
		\reactionSen
	\fi
\fi	

\ifx\microinteractionstabular\undefined
\else
	\subsection*{Übersicht über die Atomaren Interaktionen}
	\microinteractionstabular
\fi

\ifx\animations\undefined
\else
	\subsubsection*{Animationen}
	\animations
\fi

\subsection*{Hinweise zur Gestaltung der Interaktion}
\designnotes

% -------------------------------
% WANN
% -------------------------------

\section*{Wann}

\subsection*{Geeigneter Nutzungskontext}
\validcontext

\subsubsection*{Zeit}
\checkbox{\simultaneously} gleichzeitige Nutzung der beteiligten Geräte \\
\checkbox{\sequentially} sequentielle Nutzung der beteiligten Geräte

\subsubsection*{Ort}
\checkbox{\private} privat \\
\checkbox{\semipublic} halb-öffentlich \\
\checkbox{\public} öffentlich \\
\checkbox{\stationary} stationär \\
\checkbox{\onthego} unterwegs 

\subsubsection*{Körperhaltung der Benutzer}
\checkbox{\leanback} Lean-Back \\
\checkbox{\leanforward} Lean-Forward 

\subsubsection*{Teilnehmer}
\checkbox{\single} Einzelnutzer \\
\checkbox{\collaboration} Kollaboration

\subsubsection*{Anordnung zwischen Sender und Empfänger}
\checkbox{\facetoface} Face-To-Face \\
\checkbox{\sidetoside} Side-To-Side

\subsection*{Abzuratender Nutzungskontext}
\notvalidcontext

\subsection*{Geräteklassen}
\devicetabular


% -------------------------------
% WARUM
% -------------------------------

\section*{Warum}
\checkbox{\established} Bewährtes Interaction Pattern \\
\checkbox{\candidate} Interaction Pattern Kandidat: 
\checkbox{\realizable} realisierbar oder
\checkbox{\futuristic} futuristisch

\subsection*{Verwandte Patterns}
\otherpatterns

\subsection*{State of the Art}
\stateoftheart

\subsection*{Checkliste: Entspricht die Interaktion der Definiton einer "Blended Interaction"?}
\checkbox{\designprinciples} Werden die Designprinzipien berücksichtigt?
\begin{itemize}
\item[-] Die Interaktion greift eine Metapher aus der physikalischen Welt auf.
\item[-] Die Interaktion kann in einer Kollaboration ausgeführt werden.
\item[-] Die Interaktion unterstützt einen Workflow/eine Aufgabe.
\item[-] Die Interaktion findet in einer physikalischen Umgebung statt.
\end{itemize} 

\checkbox{\imageschemata} Image Schema/ta liegen zu Grunde.
\begin{itemize}
\writeifexists{\imageSchemaVoid}{Keine Image Schemata vorhanden.}
\writeifexists{\imageSchemaObject}{Object}
\writeifexists{\imageSchemaSubstance}{Substance}
\writeifexists{\imageSchemaCenterPeriphery}{Center-Periphery}
\writeifexists{\imageSchemaContact}{Contact}
\writeifexists{\imageSchemaFrontBack}{Front-Back}
\writeifexists{\imageSchemaLocation}{Location}
\writeifexists{\imageSchemaNearFar}{Near-Far}
\writeifexists{\imageSchemaPath}{Path}
\writeifexists{\imageSchemaSourcePathGoal}{Source-Path-Goal}
\writeifexists{\imageSchemaUpDown}{Up-Down}
\writeifexists{\imageSchemaLeftRight}{Left-Right}
\writeifexists{\imageSchemaContainer}{Container}
\writeifexists{\imageSchemaContent}{Content}
\writeifexists{\imageSchemaInOut}{In-Out}
\writeifexists{\imageSchemaSurface}{Surface}
\writeifexists{\imageSchemaMerging}{Merging}
\writeifexists{\imageSchemaSplitting}{Splitting}
\writeifexists{\imageSchemaMomentum}{Momentum}
\writeifexists{\imageSchemaSelfMotion}{Self-Motion}
\writeifexists{\imageSchemaBigSmall}{Big-Small}
\writeifexists{\imageSchemaFastSlow}{Fast-Slow}
\writeifexists{\imageSchemaPartWhole}{Part-Whole}
\end{itemize}

\checkbox{\realworld} Die real-weltlichen Kenntnisse des Menschen werden berücksichtigt.
\begin{itemize}
\item[-] \checkbox{\realworldNaivePhysic} Naive Physik
\item[-] \checkbox{\realworldBodyAwareness} Body Awareness and Skills
\item[-] \checkbox{\realworldEnvironmentAwareness} Environmental Awareness and Skills
\item[-] \checkbox{\realworldSocialAwareness} Social Awareness and Skills
\end{itemize}

\checkbox{\metaphor} Es ist eine natürliche Interaktion. Metapher/Assoziation: \metaphordesc

% -------------------------------
% TECHNISCHES
% -------------------------------

\ifx\requiredTechnologies\undefined	
\else
	\section*{Technisches}

	\subsection*{Benötigte Technologien}
	\requiredTechnologies

	\ifx\implementation\undefined	
	\else
		\subsection*{Implementierungshinweise}
		\implementation
	\fi
\fi


% -------------------------------
% SONSTIGES
% -------------------------------

\section*{Sonstiges}

\subsection*{Autor/en}
\authors

\subsection*{Versionshistorie}
Erstelldatum: \dateofcreation \\
Letzte Änderung am: \versionhistory


\ifx\comments\undefined	
\else
	\subsection*{Kommentare}
	\comments
\fi

\ifx\questions\undefined	
\else
	\subsection*{Offene Fragen}
\questions
\fi

\listoffigures

\printbibliography

\clearpage

\printglossaries

\end{document}

